
desenvolvimento do projeto
As reuni�es continuaram sendo realizadas no decorrer do desenvolvimento. Pelo fato de a equipe da UFMS ser menor, as reuni�es eram mais informais, geralmente entre o respons�vel pelo projeto na institui��o e um aluno envolvido. No ICMC foi mantido o planejamento e a realiza��o das reuni�es. Atas continuaram a ser elaboradas, de forma a divulgar entre todos os membros o que havia sido discutido e as decis�es tomadas (por exemplo, posso anexar o email da Renata de 31/01/2005, contendo o planejamento da reuniao e a ata do Luciano contendo a ata).   

\subsubsection{Planejamento e Gerenciamento do Projeto}

No in�cio do desenvolvimento do projeto algumas planilhas foram geradas com o objetivo de servirem como base para o planejamento inicial e o gerenciamento das atividades do projeto. Desse modo, foram geradas 3 planilhas contendo o relacionamento entre as metas do projeto e as pessoas respons�veis, o relacionamento entre metas e prazos para cumprimento e o relacionamento entre metas, prazos para cumprimento e artefatos que deveriam ser elaborados. As atividades cumpridas pelos participantes para o cumprimento das metas eram discutidas em reuni�es do projeto. Eventualmente, era solicitado (por email enviado � lista) que os participantes indicassem as atividades que estavam cumprindo e os resultados que estavam sendo obtidos (por exemplo, posso anexar o email do Luciano de 31/01/2005 - emails estao no gmail tambem).         

Observou-se que as planilhas foram �teis na atribui��o inicial das metas aos participantes e na defini��o de prazos e de artefatos. Durante o desenvolvimento do projeto, ao inv�s de atualizar as planilhas optou-se por utilizar um sistema de gerenciamento online, no caso, o sistema NetOffice. As pessoas envolvidas foram motivadas a acrescentar informa��es referentes ao desenvolvimento de suas atividades, de forma a evitar que um dos membros precisasse solicitar as informa��es por email e fazer as atualiza��es, conforme estava ocorrendo anteriormente. Apesar de a equipe ter sido treinada para usar a ferramenta e a import�ncia em utiliz�-la ter sido enfatizada, poucas informa��es de fato foram registradas. Grande parte das informa��es continuou a ser registrada por um dos membros do projeto, de acordo com as atividades e os resultados apresentados nas reuni�es. De forma similar ao que acontece em ambiente industrial, notou-se que n�o houve entre os estudantes a cultura de documentar, passo a passo, as atividades que cumprem. No geral, apenas os resultados finais s�o apresentados, geralmente na forma de um artefato. 

A alternativa adota entao foi continuar mantendo registros utilizando-se o sistema NetOffice, de forma centralizada, ou seja, um dos membros atualizava os registros de acordo com o que era discutido nas reuni�es. Um exemplo de uso da ferramenta contendo informa��es de gerenciamento do projeto � apresentado no Anexo X (posso colocar um grafico gerado atividades X prazos). 
     

