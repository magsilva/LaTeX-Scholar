\section{Caracteriza��o do processo almejado} \label{alvo}

De acordo com Humphrey, nesta etapa deve-se identificar os objetivos a serem alcan�ados em rela��o ao processo que est� sendo proposto, bem como a estrutura e os principais elementos do processo.

Para a apresenta��o do processo padr�o foi utilizada a estrutura indicada na norma ISO/IEC 12207, em que os processos s�o divididos em processos fundamentais, processos de apoio e processos organizacionais. Assim, a estrutura geral do processo almejado p�de ser ent�o definida: 

\begin{description}
\item [1. Processos Fundamentais:]   \sffamily Processo de aquisi��o,  Processo de inicia��o,    Processo de desenvolvimento,   Processo  de opera��o,  Processo  de manuten��o.
 
\normalfont
\item [2. Processos de Apoio:]  \sffamily Processo de documenta��o,  Processo de ger�ncia de configura��o,  Processo de garantia da qualidade,  Processo de verifica��o,  Processo de valida��o,  Processo de revis�o,  Processo de resolu��o de problemas,  Processo de revis�o sistem�tica,  Processo de prepara��o de documentos cient�ficos,  Processo de elabora��o de m�dulos educacionais,  Processo de {\it postmortem} e Processo de transfer�ncia tecnol�gica.

\normalfont
\item [3. Processos Organizacionais:]  \sffamily Processo de treinamento, Processo de ger�ncia, Processo de infra-estrutura, Processo de melhoria, Processo de planejamento, Processo de divulga��o, Processo de comunica��o, Processo de coordena��o e Processo de estabelecimento de parceria universidade-empresa.

\end{description}

\normalfont